% Czym jest cykl w tworzeniu oprogramowania?
Cyklem w tworzeniu oprogramowania nazywamy proces wprowadzenia zmiany.
Oprogramowanie realizuje cele twórców w różnych dziedzinach, od sterowania statkiem powietrznym
do obsługi płatności i wyświetlania obrazu, zatem definicja cyklu może różnić się od stosowanej technologii.

W zasadzie, występują trzy podstawowe elementy cyklu:
\begin{itemize}
    \item Określenie celu zmiany
    \item Wprowadzenie zmiany
    \item Testowanie zmiany
\end{itemize}

Czas całego cyklu to suma czasu wykonania tych czynności.
Złożony cel nierzadko bywa rozkładany przez programistów na mniejsze cele, wobec
czego osiągnięcie celu skłąda się z wielu cykli.
Programiści często stosują tą metodę, by ograniczyć obciążenie poznawcze, które przyczynia się do
wprowadzenia błędów wraz ze zmianą.

% Dlaczego istotny jest jego czas?

Czas wymagany na wprowadzenie zmiany w tworzeniu oprogramowania bezpośrednio wiąże się z kosztem
wprowadzenia zmiany.
Czas programisty jest ograniczony, wobec nie jest bez znaczenia ile czasu zajmie wprowadzenie
tej czy innej zmiany.
Czas cyklu jest przez to bezpośrednio związany z kosztem wprowadzenia zmiany bądź kosztem alternatywnym,
t.j. potencjalnymi zyskami z pracy programisty nad innym zadaniem w poświęconym czasie.

